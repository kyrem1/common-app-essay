\documentclass[12pt]{article}

%%%%%%%%%%%%%%%%%%%%%%%%%%%%%%%%%%-- Settings --%%%%%%%%%%%%%%%%%%%%%%%%%%%%%%%%%%%%%%%%%%%
\usepackage[english]{babel}

% - Margin - 1 inch on all sides
\usepackage[letterpaper]{geometry}
\usepackage[utf8]{inputenc}
\geometry{top=1.0in, bottom=1.0in, left=1.0in, right=1.0in}

% - Double Spacing -
\usepackage{setspace}
\doublespacing
\setstretch{2.00}

% % FancyHDR settings
% \setlength{\headheight}{15.2pt}
% \usepackage{fancyhdr}
% \pagestyle{fancy}
% \fancyhf{}

% Indent First paragraph
\usepackage{indentfirst}

% Other Packages
\usepackage{outlines}

% Title Settings
\title{
    \vspace{2in}
    \textmd{\textbf{Common Application - Essay Portion}}\\
    \normalsize\vspace{0.1in}\small{Due\ on\  August, 1\ at 11:59pm}\\
    \vspace{3in}
}
\author{James Harbour}

\renewcommand{\footnoterule}{%
  \kern -3pt
  \hrule width \textwidth height 0.5pt
  \kern 2pt
}

%%%%%%%%%%%%%%%%%%%%%%%%%%%%%%%%%%-- Assignment --%%%%%%%%%%%%%%%%%%%%%%%%%%%%%%%%%%%%%%%%%%%

  % Common Application : general essay
  % Words Min: 250
  % Words Max: 650

%%%%%%%%%%%%%%%%%%%%%%%%%%%%%%%%%%-- TODOLIST --%%%%%%%%%%%%%%%%%%%%%%%%%%%%%%%%%%%%%%%%%%%
  % TODO Finish Transitions
  % TODO AL to FL move introduction
  % TODO CUT DOWN ON WORD COUNT : 800 words as of July 20th, 2020; needs to be 650
  % TODO Weave S1/S2 USF and Profs into me wanting to become mathematician
%%%%%%%%%%%%%%%%%%%%%%%%%%%%%%%%%%-- Document --%%%%%%%%%%%%%%%%%%%%%%%%%%%%%%%%%%%%%%%%%%%

% Begin Document
\begin{document}
\maketitle
\pagebreak
 \begin{center}

   \emph{Some students have a background, identity, interest, or talent that is so meaningful they believe their application would be incomplete without it. If this sounds like you, then please share your story.}

 \end{center}
\raggedright\setlength{\parindent}{0.5in}

% Hook
Upon finally understanding the ingenious proof of Cantor’s theorem, I once again felt that familiar sense of wonder towards the giants whose shoulders I now recklessly aspire to stand upon. Always following this moment of reverence, I experience the ineffable pleasure of rightly viewing mathematical beauty.

\begin{outline}[enumerate]
  \1 Elementary School \& into 6th Grade
    \2 Like many similar children plagued with an unquenchable curiosity, I was bullied throughout my elementary school years. Moreover, I had never experienced any academic challenge, and so I transferred to a private school that, in comparison to my previous experiences, appeared almost like an intellectual oasis.
  \1 6th, 7th \& 8th Grade
    \2 However, I quickly came to know that accompanying such an environment was the reality that I was no longer the \emph{smart} kid. For the first time I felt what it was like to be at the bottom of the totem pole, and I realized that I would have to work my way up by struggling at every step.
    \2 After over a year of being the only member of the math team that wasn't ahead in math, I decided to contact the head of the department, Dr. Bernard, to inquire about skipping from pre-Algebra, my then-current class, to Algebra I in the middle of the school year.
    \2 Dr. Bernard told me he would accept my request if I could obtain a near perfect score on the Algebra I midterm.
    Unfazed by this monumental task, I spent the entirety of my Christmas break learning an entire semester of material. To the surprise of both my friends and family, I succeeded in obtaining the purported score and finally stood on the same level as my peers in the math team; thereafter, I realized that I actually enjoyed the sense of accomplishment that arose from mastering material at such a fast pace. Hence, I took geometry during the following summer and snowballed my mathematics level even further.
  \1 9th \& 10th Grade
    \2 Due to financial reasons, for 9th and 10th grade I transferred to Vestavia Hills High School (VHHS): a school with one of the top math teams in the nation. Since I attend the same school as my VHHS peers during middle school, I was at a disadvantage due to my lack of competition experience as well as problem-solving skills. Being helplessly behind my peers in mathematics knowledge despite being two classes ahead of them helped me realize the importance of grinding, not just speedrunning. Following this realization, I plunged myself into learning as much mathematics as possible during the time. Yet, as time passed, I became increasingly dissatisfied with the base level of formula-bashing used in local competitions and often was distracted by the proofs behind facts rather than their applications.
    % TODO Transition btw sections.
  \1 11th \& 12th Grade
    \2 -- Introduce paragraph with Move to FL and Transfer to Steinbrenner -- % TODO
      \3 Abruptly, I found out that my father had taken a job promotion that would require our family to move to Florida, and thus I would have to transfer schools. Following inquiries with both district and county level bureaucrats, I quickly learned that I would no longer have access to the wonderful academically competitive atmosphere that the VHHS math team provided as similar programs simply did not exist in the area.
      \3 Although distraught over the loss of such a uniquely enriching environment, a silver lining quickly arose in the form of the county's exciting dual enrollment policies. As I had already taken every mathematics course offered by my new high school, I realized that in my remaining two years of high school I could pursue my mathematics education much further than I could have ever dreamed to as a high-schooler; henceforth, I began my dual enrollment at the University of South Florida.
    \2 -- Dual Enrollment with USF --
      \3 Under the county's policy, I could take up to three mathematics courses each semester, and thus I enrolled in Calculus 3, Linear Algebra, and Bridge to Abstract Mathematics for my first semester. Despite feeling mild apprehension regarding my ability to manage taking three college courses on top of taking all Advanced Placement courses at my high school, I ultimately plunged headlong into the heaviest courseload I could at the time by taking the maximum amount of classes allowed by the county. It did not take long for me to understand the delightful fact that the sheer quantity and quality of information I would learn in my courses at USF would dwarf that of even the most informative of my high school classes. Moreover, this understanding quickly morphed into an appreciation and respect for Dr. Nagle, the professor that taught both my Linear Algebra section, laden with proofs and abstraction, and my Bridge to Abstract Mathematics section, a class whose sole purpose is to teach undergraduates to write proofs whilst gaining an appreciation of their necessity. Every lecture I was captivated by the sheer amount of thought required to stand doubtless in front of trivialities such as Euclidean division and the Fundamental Theorem of Arithmetic, whose truth every grade schooler is conditioned to accept without contest.
    \2 Ending
      \3 As I took more difficult courses in the following semesters and subsequently formed more connections with my professors, I gradually realized that my previous obsession with proofs and increasing rigor signified that I had been thinking like a mathematician all along. Moreover, in observing them and reading their publications, I finally understood that I too wanted to follow in those footsteps, to someday create something new. In order to do so, I will have climb upon the shoulders of all the giants before me, respectfully stopping at each step to experience the beauty of their ideas. I now desire to become the shoulders that someday are stood upon to peer further into the unknown than I will in my lifetime and, while doing so, provide that same sense of wonder to all those who will figuratively stand above me.


\end{outline}



% Expression of Ambition [kinda vague, tho might make a good ending]




\end{document}
