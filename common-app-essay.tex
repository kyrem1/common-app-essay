\documentclass[12pt]{article}

%%%%%%%%%%%%%%%%%%%%%%%%%%%%%%%%%%-- Settings --%%%%%%%%%%%%%%%%%%%%%%%%%%%%%%%%%%%%%%%%%%%
\usepackage[english]{babel}

% - Margin - 1 inch on all sides
\usepackage[letterpaper]{geometry}
\usepackage[utf8]{inputenc}
\geometry{top=1.0in, bottom=1.0in, left=1.0in, right=1.0in}

% - Double Spacing -
\usepackage{setspace}
\doublespacing
\setstretch{2.00}

% % FancyHDR settings
% \setlength{\headheight}{15.2pt}
% \usepackage{fancyhdr}
% \pagestyle{fancy}
% \fancyhf{}

% Indent First paragraph
\usepackage{indentfirst}

% Title Settings
\title{
    \vspace{2in}
    \textmd{\textbf{Common Application - Essay Portion}}\\
    \normalsize\vspace{0.1in}\small{Due\ on\  August, 1\ at 11:59pm}\\
    \vspace{3in}
}
\author{James Harbour}

\renewcommand{\footnoterule}{%
  \kern -3pt
  \hrule width \textwidth height 0.5pt
  \kern 2pt
}

%%%%%%%%%%%%%%%%%%%%%%%%%%%%%%%%%%-- Assignment --%%%%%%%%%%%%%%%%%%%%%%%%%%%%%%%%%%%%%%%%%%%

%%%%%%%%%%%%%%%%%%%%%%%%%%%%%%%%%%-- Document --%%%%%%%%%%%%%%%%%%%%%%%%%%%%%%%%%%%%%%%%%%%

% Begin Document
\begin{document}
\maketitle
\pagebreak
 \begin{center}

   \emph{Some students have a background, identity, interest, or talent that is so meaningful they believe their application would be incomplete without it. If this sounds like you, then please share your story.}
 \end{center}
\raggedright\setlength{\parindent}{0.5in}


Upon finally understanding the ingenious proof of Cantor’s theorem, I once again felt that familiar sense of wonder towards the giants whose shoulders I now recklessly aspire to stand upon. Always following this moment of reverence, I experience the ineffable pleasure of rightly viewing mathematical beauty.

I desire to become the shoulders that someday are stood upon to peer further into the unknown than I will and, while doing so, provide that same sense of wonder to all those figuratively standing above me.

Every lecture I was captivated by the sheer amount of thought required to stand doubtless in front of trivialities such as Euclidean Division and the Fundamental Theorem of Arithmetic, whose truth every grade schooler is conditioned to accept without contest.

\end{document}
