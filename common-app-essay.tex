\documentclass[12pt]{article}

%%%%%%%%%%%%%%%%%%%%%%%%%%%%%%%%%%-- Settings --%%%%%%%%%%%%%%%%%%%%%%%%%%%%%%%%%%%%%%%%%%%
\usepackage[english]{babel}

% - Margin - 1 inch on all sides
\usepackage[letterpaper]{geometry}
\usepackage[utf8]{inputenc}
\geometry{top=1.0in, bottom=1.0in, left=1.0in, right=1.0in}

% - Double Spacing -
\usepackage{setspace}
\doublespacing
\setstretch{2.00}

% % FancyHDR settings
% \setlength{\headheight}{15.2pt}
% \usepackage{fancyhdr}
% \pagestyle{fancy}
% \fancyhf{}

% Indent First paragraph
\usepackage{indentfirst}

% Other Packages
\usepackage{outlines}

% Title Settings
\title{
    \vspace{2in}
    \textmd{\textbf{Common Application - Essay Portion}}\\
    \normalsize\vspace{0.1in}\small{Due\ on\  August, 1\ at 11:59pm}\\
    \vspace{3in}
}
\author{James Harbour}

\renewcommand{\footnoterule}{%
  \kern -3pt
  \hrule width \textwidth height 0.5pt
  \kern 2pt
}

%%%%%%%%%%%%%%%%%%%%%%%%%%%%%%%%%%-- Assignment --%%%%%%%%%%%%%%%%%%%%%%%%%%%%%%%%%%%%%%%%%%%

%%%%%%%%%%%%%%%%%%%%%%%%%%%%%%%%%%-- Document --%%%%%%%%%%%%%%%%%%%%%%%%%%%%%%%%%%%%%%%%%%%

% Begin Document
\begin{document}
\maketitle
\pagebreak
 \begin{center}

   \emph{Some students have a background, identity, interest, or talent that is so meaningful they believe their application would be incomplete without it. If this sounds like you, then please share your story.}
 \end{center}
\raggedright\setlength{\parindent}{0.5in}

% Hook
Upon finally understanding the ingenious proof of Cantor’s theorem, I once again felt that familiar sense of wonder towards the giants whose shoulders I now recklessly aspire to stand upon. Always following this moment of reverence, I experience the ineffable pleasure of rightly viewing mathematical beauty.

\begin{outline}[enumerate]
  \1 7th \& 8th Grade
    \2 -- Introduce paragraph with Transfer to Altamont --
    \2 After over a year of being the only member of the math team that wasn't ahead in math, I decided to contact the head of the department, Dr. Bernard, to inquire about skipping from pre-Algebra, my then-current class, to Algebra I in the middle of the school year.
    \2 Dr. Bernard told me he would accept my request if I could obtain a near perfect score on the Algebra I midterm. Unfazed by this monumental task, I spent the entirety of my Christmas break learning an entire semester of material. To the suprise of both my friends and family, I succeeded in obtaining the purported score and finally stood on the same level as my peers in the math team; thereafter, I realized that I actually enjoyed the sense of accomplishment that arose from mastering material at such a fast pace. Hence, I took geometry during the following summer and snowballed my mathematics level even further.
  \1 9th \& 10th Grade
    \2 -- Introduce paragraph with Transfer to Vestavia --
    \2 Helplessly behind my peers in mathematics knowledge despite being two classes ahead of them
    \2 --> Helped me realize the importance of grinding, not just speedrunning.
    \2 I plunged myself into learning as much mathematics as possible during the time.
    \2 Realized I wasn't satisfied with the base level of formula-bashing used in local competitions and often was distracted by the proofs behind facts rather than their application.
  \1 11th \& 12th Grade
    \2 -- Introduce paragraph with Move to FL and Transfer to Steinbrenner --
    \2 -- Dual Enrollment with USF --
    \2 --> My obsession with proofs and increasing rigor helped me immensively with the proof-based courses I took and am taking at USF.
\end{outline}

% Expression of Ambition [kinda vague tho]
I desire to become the shoulders that someday are stood upon to peer further into the unknown than I will and, while doing so, provide that same sense of wonder to all those figuratively standing above me.


Every lecture I was captivated by the sheer amount of thought required to stand doubtless in front of trivialities such as Euclidean Division and the Fundamental Theorem of Arithmetic, whose truth every grade schooler is conditioned to accept without contest.

\end{document}
